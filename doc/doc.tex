\documentclass[11pt]{article}
\usepackage[a4paper, margin=2.5cm]{geometry}
\usepackage[document]{ragged2e}
% \usepackage[utf8]{inputenc}
\usepackage{listings}
\usepackage{setspace}
\usepackage{graphicx}
\usepackage{caption}
\usepackage{subcaption}
\usepackage{setspace}
\usepackage{graphicx}
\graphicspath{ {img/} }
\title{X3Dom Exporter}
\date{\today}
\author{Gabriele Vassallo}


\begin{document}
    % \begin{titlepage}
    %     \begin{center}

        %Title 
        \maketitle 
        \vfill
        \center \includegraphics[height=10.0cm]{x3dom-logo}
       % \end{center}
%     \end{titlepage}

\clearpage 
\tableofcontents
\clearpage 

\section{Introduzione}
% \subsection{X3Dom}
\begin{flushleft}
X3Dom e' un framework per integrare e manipolare scene (X)3D come fossero elementi DOM in HTML5, i quali sono renderizzati attraverso un plugin X3D, Falsh/Stage3D o WebGL. \\
Per cui, non conoscendo nessuna delle tecnologie sopracitate, e' possibile realizzare una scena web 3D attraverso l'utilizzo di un liguaggio dichiarativo quale l'X3D. \\
La direttiva che dichiara l'inizio della scena 3D e' il tag X3D ; da qui in avanti si utilizza tale linguaggio per popolare la scena con diversi elementi ed animazioni. 
Di seguito un'esempio : 

\vspace{10mm}
\small \lstinputlisting[language=html]{res/simple.html}
\center \includegraphics[height=9cm]{simple}
\end{flushleft}
\clearpage

\begin{flushleft}
X3DOM utilizza pagine (X)HTML per incapsulare dati X3D e file X3D opzionali ai quali potrebbero fare riferimento le parti incapsulate. I file X3D, infatti, possono fare riferimento ad altri file dello stesso tipo dando vita ad una gerarchia di indicizzazione. 
In questo modo la conversione X3D, X3Dom potrebbe sembrare molto semplice, se non fosse che, in questo momento, l'implementazione degli elementi X3D e' completa solo al 
60\% circa. 
Del quasi 40\% dei nodi mancanti fanno parte alcuni dei nodi piu' importanti tra cui filtri booleani o essenziali sensori di click. 
Per ovviare tale problema, il programmatore X3Dom, operando in ambiente Web, potrebbe sfruttare i meccanismi di Event-Trigger di HTML come, ad esempio, l'evento 'onclick' del javascript. 
\newline
Poiche' ho dovuto operare in un contesto diverso, ossia in un'ottica di conversione tra questi due formati, ho ritenuto importante utilizzare alcuni dei nodi mancanti; a tal proposito ho sfruttato le API di X3Dom per implementare alcune di queste lacune. 
\newline 
In particolare ho realizzato in X3Dom i seguenti nodi : 
    \begin{itemize}
        \item Booleans :
            \begin{itemize}
                \item BooleanFilter
                \item BooleanToggle
            \end{itemize}
        \item Sequencers
            \begin{itemize}
                \item BooleanSequencer
                \item IntegerSequencer 
            \end{itemize}
        \item Triggers
            \begin{itemize}
                \item BooleanTrigger
                \item TimeTrigger
            \end{itemize}
        \item TouchSensor
    \end{itemize}
\end{flushleft}
\end{document}
